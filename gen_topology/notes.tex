\documentclass[12pt,reqno]{amsart}

\usepackage{graphicx}

\usepackage{amssymb}
\usepackage{amsthm}
\usepackage{mathrsfs}
\theoremstyle{plain}

\newtheorem*{thm*}{Theorem}
%% this allows for theorems which are not automatically numbered

\newtheorem{thm}{Theorem}
\newtheorem{lem}{Lemma}
\theoremstyle{definition}
\newtheorem{defn}{Definition}
\newtheorem{eg}{Example}
\newtheorem{rem}{Remark}
\usepackage{lineno}

%% The above lines are for formatting.  In general, you will not want to change these.


\title{General Topology}
\author{Devansh Tripathi}

\begin{document}

\begin{abstract}
    We shall learn some general topology.
\end{abstract}
\maketitle
\begin{defn}
    Induced metric: A metric which is derived from a norm. A normed space is a special metric space whose metric is derived from a norm.
\end{defn}
\begin{eg}
    $\mathscr{C}[a,b]$: Set of all bounded continuous real function on a closed interval form the normed space with norm defined as $$ \|f\| = \int_a^b |f(x)|~dx \hspace{0.35cm}\text{or,} \hspace{0.5cm} \|f\| = \sup{|f(x)|}$$ and the induced metric is $$ \|f - g \| = \int_a^b |f(x) - g(x)| ~ dx \hspace{0.35cm} \text{or,} \hspace{0.5cm} \|f - g\| = \sup{|f(x) - g(x)|} $$
\end{eg}
\begin{defn}
    Distance of a point $x$ from a set $A$: $$ d(x, A) = \inf\{d(x,a) \mid \forall a \in A \} $$
    Diameter of the set: $$ d(A) = \sup \{d(a_1, a_2) \mid \forall a_1, a_2 \in A\} $$
\end{defn}
\begin{defn}
    Bounded mapping: A mapping $f$ of a non-empty set into a metric space is said to be bounded if its range is bounded i.e. $\exists M \in \mathbb{R} \text{ such that } |f(x)| \leq M $ 
\end{defn}
\begin{eg}
    A pseudo metric which is not a metric
    $$ f,g \in \mathbb{R}^2 \text{ and } d(f,g):= \text{difference between their $x$ coordinates} $$    
\end{eg}
\begin{defn}
    Interval: A set $A \subset \mathbb{R}$ is an interval if $$ \forall x, y \in A \text{ and } \forall t \in \mathbb{R} \colon x \leq t \leq y \implies t \in A $$ 
\end{defn}
\begin{thm}
    Union of intervals with non empty intersection is an interval.    
\end{thm}
\begin{proof}
    Let $\{I_i\}$ be the set of interval and $a \in \cap_i I_i$.\\
    Proof Idea: Take any two points in the union and show that they contains every point in between them (take general point and show that it will belong to the union).\\
    Let $x,y \in \cup_i I_i$ and let $t \in \mathbb{R} \colon x \leq t \leq y $ then there are following possiblities:\\
    $t < a$, \\$ t = a$ or, \\ $t > a$.\\ All are trivial to show that they lie in union.
\end{proof}
\section{Topological Spaces}
\begin{defn}
    Topology: A topology on a set $X$ is a collection $\mathcal T$ of subsets of $X$ having the following properties:
\begin{itemize}
    \item $\phi$ and $X$ are in $\mathcal T$.
    \item The union of the elements of any subcollection of $\mathcal T$ is in $\mathcal T$.
    \item The intersection of the elements of any finite subcollection of $\mathcal T$ is in $\mathcal T$.
\end{itemize}
\end{defn}
A set $X$ with topology $\mathcal T$ is called an topological space ($X,\mathcal T$).
\begin{defn}
    Open set of $X$: For the topological space $(X, \mathcal T)$, a subset $U$ of $X$ is an open set of $X$ if $U$ belongs to the collection $\mathcal T$.
\end{defn}
\begin{eg}
    Discrete Topology: If $X$ is any set then collection of all subsets of $X$ is a topology on $X$, called {\bf discrete topology.}
\end{eg}
\begin{eg}
    Indiscrete or trivial topology: The topology consisting of only $\phi$ and the whole set $X$ is called {\bf trivial topology.}
\end{eg}
\begin{eg}
    Finite complement topology: Let $X$ be a set and $\mathcal{T}$ be the collection of all subset $U$ of $X$ such that $X - U$ is either finite or $X$. Then $\mathcal{T}$ is called {\bf finite complement topology}. (This topology is consists of subset of $X$ whose complement is either finite or $X$.)
\end{eg}
\begin{proof}
    Let $\{U_{i}\}$ be the indexed family of subsets of $X$ belongs to $\mathcal{T}$. $\phi$ and $X$ are obviously there. Assume each $\bigcup\limits_i U_i$ is non-empty (trivial for empty case): 
    $$X - \bigcup\limits_i U_i = \bigcap\limits_i(X - U_i)$$
    Since each $U_i$ is in $\mathcal{T}$, $X - U_i$ is finite. and $\bigcap\liminf_i X - U_i$ is contained in every $X - U_i$ hence it is finite.\\ \\
    To show $\bigcap\limits_i^n X - U_i$ is in $\mathcal{T}$, 
    $$ X - \bigcap\limits_i^n U_i = \bigcup\limits_i^n (X - U_i) $$
    Rhs is finite union of finite sets hence it is finite.
\end{proof}
\begin{eg}
    Let $X$ be set and $\mathcal{T}_c$ be the collection of all subsets $U$ of $X$ such that $U^c$ is either countable or all of $X$. Then $\mathcal{T}_c$ is a topology of $X$. 
\end{eg}
\begin{proof}
    $\phi$ and $X$ are trivial inside $\mathcal{T}_c$. Let $U_i$ be the indexed family of subsets of $X$. Assume $\bigcup\limits_i U_i$ is non-empty (trivial for empty case). To show that $\bigcup\limits_i U_i$ is in $\mathcal{T}_c$
    $$ X - \bigcup\limits_i U_i = \bigcap\limits_i (X - U_i)$$
    Since, $X - U_i$ is countable for each $i$ and $\bigcap\limits_i (X - U_i)$ is in $U_i$ for each $i$. Hence, $\bigcap\limits_i (X - U_i)$ is countable. \\
    To show that $\bigcap\limits_i U_i$ is in $\mathcal{T}_c$, use the same argument as last example and the fact that finite union of countable sets is countable.
\end{proof}
\begin{defn}
    Finer or strictly finer topology: For a set $X$, if $\mathcal{T}$ and $\mathcal{T}'$ are two topologies on $X$ such that $\mathcal{T} \subset \mathcal{T}'$ then we say $\mathcal{T}'$ is {\bf finer} than $\mathcal{T}$ and if $\mathcal{T}'$ properly contains $\mathcal{T}$ then we say it's {\bf strictly finer}. Then $\mathcal{T}$ is called {\bf coarser} than $\mathcal{T}'$ or, {\bf strictly coarser} if it is contained in $\mathcal{T}'$ properly.
\end{defn}
\begin{lem}
    Let $\mathcal{B}$ and $\mathcal{B}'$ be the bases for the topologies $\mathcal{T}$ and $\mathcal{T}'$, respectively, on $X$. Then following statements are equivalent:
    \begin{itemize}
        \item $\mathcal{T}'$ is finer than $\mathcal{T}$.
        \item For each 
    \end{itemize}
\end{lem}
\begin{defn}
    Comparable: We say $\mathcal{T}$ is comparable with $\mathcal{T}'$ if either $\mathcal{T} \subset \mathcal{T}'$ or $\mathcal{T}' \subset \mathcal{T}$.
\end{defn}
\section{Basis for a Topology}
\begin{defn}
    If $X$ is a set, a {\bf basis} for a topology on $X$ is a collection $\mathcal{B}$ of subsets of $X$ (called basis elements) such that
    \begin{itemize}
        \item For each $x \in X$, there is atleast one basis element $B \in \mathcal{B}$ such that $x \in B$.
        \item If $x$ belongs to the intersection of two basis elements $B_1$ and $B_2$, then there is $B_3 \in \mathcal{B}$ such that $x \in B_3 \subset B_1 \cap B_2$. 
    \end{itemize}
\end{defn}
We define a topology $\mathcal{T}$ generated by $\mathcal{B}$ as: A subset $U$ of $X$ is said to be open in $X$ (e.g. an element of topology on $X$) if for all $x \in U$, there is a basis element $B \in \mathcal{B}$ such that $x \in B \subset U$.

\begin{rem}
    Each element of the basis is an element of the topology. 
\end{rem}
\begin{eg}
    If $X$ is any set then the collection of all one element subsets of $X$ is a basis for the discrete topology on $X$.(Power set of $X$). 
\end{eg}
\begin{proof}
    Trivial to see. (Caution: Do not take element of the topology on $X$. For basis, condition is on the elements of the set $X$ hence take element of $X$ and then check basis conditions.)
\end{proof}
\begin{lem}
    The collection $\mathcal{T}$ generated by the basis $\mathcal{B}$ is a topology.
\end{lem}
\begin{proof}
    Let the collection $\mathcal{T} = \{U_i\}_{i \in I}$. Condition for the set $U_i$ to belong to the collection is that for each $x \in U_i$ there exists an element $B \in \mathcal{B}$ and $x \in B \subset U_i$.
    \paragraph{\bf Membership of $\phi$ and $X$:}
    For $\phi$, it is vacuously true (true due to non-availability of elements in the set). For $X$, for each $x \in X$, there exists $B \in \mathcal{B}$ (by definition of basis) such that $x \in B$ and $B \subset X$.
    \paragraph{\bf Closure under arbitray union of elements.} Now, assume that $\{U_i\}_{i \in I}$ is the indexed family of subsets of $X$ which are elements of $\mathcal{T}$. We need to show that $\bigcup\limits_{i \in I} U_i \in \mathcal{T}$. For each $x \in \bigcup\limits_{i} U_i \implies x \in U_i$ for some $i$ and $U_i \in \mathcal{T} \implies \exists B \in \mathcal{B} \text{ such that } x \in B \subset U_i$. This completes the argument.
    \paragraph{\bf Closure under finite intersection.} We need to show that $\bigcap\limits_{i = 0}^{n} U_i \subset \mathcal{T}$. For each $x \in \bigcap\limits_{i = 0}^{n} U_i$
    $$ x \in U_i ~~\forall i \implies \exists B_i \in \mathcal{B} ~~\forall i \in \{0,1,\dots n\}$$
    Since, $x \in \bigcap\limits_{i = 0}^{n} B_i$ and $B_i~'s$ are basis elements hence by definition of basis, $\exists B' \in \mathcal{B}$ such that $x \in B' \subset \bigcap\limits_{i = 0}^{n} B_i$. Hence, $\bigcap\limits_{i = 0}^{n} U_i \subset \mathcal{T}$.
\end{proof}
\begin{lem}
    Let $X$ be a set; $\mathcal{B}$ is the set of all basis elements of the topology $\mathcal{T}$ on set $X$. Then $\mathcal{T}$ equals the collection of all unions of elements of $\mathcal{B}$.
\end{lem}
\begin{proof}
    Since each element $B$ of basis is in $\mathcal{T}$ and hence their union. For other way around, let $U \in \mathcal{T}$, then for each $x \in U~~\exists B_x \in \mathcal{B} \subset U$ hence, $ U = \bigcup\limits_{x \in U} B_x$. Therefore, each $U \in X$ is union of basis elements.
\end{proof}
\begin{rem}
    Above lemma states that every set $U$ in $X$ can be expressed as union of basis elements of the topology, however this is {\bf not unique.}
\end{rem}
\begin{lem}
    Let $X$ be an topological space. Suppose that $\mathcal{C}$ is a collection of open sets of $X$ such that for each open set $U$ of $X$ and each $x \in U$, there is an element $C$ of $\mathcal{C}$ such that $x \in C \subset \mathcal{C}$. Then $\mathcal{C}$ is a basis of the topology of $X$.
\end{lem}
\begin{proof}
    First we will prove that $\mathcal{C}$ is the basis of the topology on $X$.
    \paragraph{\bf First condition of basis:} Since $X$ is a open set of itself hence hypothesis, by for each $x \in X$ there exists $C \in \mathcal{C}$ such that $x \in C \subset \mathcal{C}$.
    \paragraph{\bf Second condition of basis:} Let $x \in C_1 \cap C_2$ for some open sets $C_1, C_2 \in \mathcal{C}$. Since $C_1, C_2$ are open in $X$ then so is $C_1 \cap C_2$ hence by hypothesis for each $x \in C_1 \cap C_2$ there exists $C_3 \in \mathcal{C}$ such that $x \in C_3 \subset C_1 \cap C_2$.
    \paragraph{\bf Topology generated by $\mathcal{C}$ equals topology of $X$.} Let $\mathcal{T}_c$ be the topology generated by $\mathcal{C}$ and $\mathcal{T}$ be a topology on $X$. Let $U \in \mathcal{T}$. For each $x \in U$, by hypothesis, there exists $C_x \in \mathcal{C}$ such that $x \in C_x \subset U$ hence $U = \bigcup\limits_{x \in U} C_x$(union of elements of $\mathcal{C}) \implies \mathcal{T} \subset \mathcal{T}_c$.\\
    Let $V \in \mathcal{T}_c \implies V = \bigcup\limits_{i \in I} C_i$ for each $C_i \in \mathcal{C}$ (by previous lemma). Since each $C_i$ are open in $X$ hence $C_i \in \mathcal{T}$ and $\mathcal{T}$ is a topology (their union will belong to $\mathcal{T}$). Hence, $V \in \mathcal{T} \implies \mathcal{T}_c \subset \mathcal{T}$. Therefore, $\mathcal{T}_c = \mathcal{T}$.
\end{proof}

\end{document}